%!TEX root = main.tex

\chapter{Introducción}

\section{Identificación del problema}
El viento es uno de los fenómenos metereológicos más comunes de las zonas costeras de Valparaíso. Sus características afectan distintos aspectos del entorno que influyen en nuestras vidas, como las condiciones del clima, la sensación térmica, algunos desastres naturales, entre otros. Esto atrae el interés de los especialistas, fomentando la investigación para lograr controlar, en cierta medida, las distintas variables que condicionan su comportamiento, de manera de poder predecir los fenómenos subyacentes a este. Más aún, el viento es ampliamente conocido por ser una fuente renovable y poco contaminante de energía, comunmente llamada energía eólica, lo que hace aún más interesante y necesario el estudio de este fenómeno metereológico.\\
En los últimos años, tanto el gobierno de Chile como la ciudadanía, han mostrado un creciente interés en las fuentes de energías renovables y con mínimo impacto ambiental, por lo que distintos proyectos en la matería han sido llevados a cabo, desde estudios de factibilidad y recopilación de datos hasta el emplazamiento de las primeras centrales de fuentes de energía límpia. El año 2014, el ministerio de energía publicó un extenso reporte acerca de la situación actual del país en materias de energías renovables, en donde se pueden ver distintas proyecciones y estimaciones de implementación de posibles plantas de generación a lo largo del país. \cite{minenergia14}\\
Cualquiera sea el estudio de las condiciones climáticas del viento para predecir su potencial energético, existen dos variables fundamentales ha considerar para obtener estas predicciones: la velocidad y la dirección del viento. Actualmente, estos datos están disponibles y son obtenidos por diversos centros metereológicos a lo largo del país, en partícular, el servicio metereológico de la Armada de Chile, que cuenta con el equipo necesario para registrar el comportamiento del viento al lo largo de las distintas épocas del año y en diferentes zonas de Chile.\\
Si bien la obtención de los datos es un paso, para poder obtener información o conocimiento relevante sobre estos (algo más que un promedio, por ejemplo), es necesario generar modelos que permitan ``explicar'' la distribución de datos y de cierta forma predecir el comportamiento futuro del viento. En la literatura, son ampliamente aceptadas la distribución de \emph{Weibull} para modelar el conjunto datos de velocidad y la distribución de Von Mises para el modelo de la dirección del viento. Ambas distribuciones probabilísticas requieren de la determinación de parámetros para que el modelo se ajuste a los datos obtenidos. Comunmente, se utilizan métodos númericos para la determinación de estos parámetros, sin embargo, estudios recientes han demostrado que la utilización de algorítmos basados en la metaheurística \emph{Particle Swarm Optimization} mejoran los parámetros obtenidos por los métodos tradicionales y por ende la aproximación del modelo al comportamiento real.\\ 

\section{Objetivos}
Aplicar métodos actuales de optimización basados en la metaheurística \emph{Particle Swarm Optimization} para el ajuste de modelos probabilísticos de dirección y velocidad del viento a los datos recopilados del viento en Valparaíso, con el fin de presentar resultados que permitan inferir información precisa acerca de las condiciones de la región para la generación de energía eólica y otras potenciales aplicaciones.

\subsection{Objetivos específicos}
\begin{enumerate}
    \item Implementar un algoritmo basado en \emph{Particle Optimization Sworm} para \textbf{encontrar} los parámetros de un modelo probabilístico que se ajusten a los datos de la velocidad del viento en Valparaíso. 
    \item Implementar un algoritmo basado en \emph{Particle Optimization Sworm} para \textbf{optimizar} los parámetros de un modelo probabilístico que se ajusten a los datos de la dirección del viento en Valparaíso.
    \item Evaluar los modelos sobre los datos del viento para validar la propuesta realizada.
\end{enumerate}

