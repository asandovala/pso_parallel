%!TEX root = main.tex

\chapter{Desarrollo de la solución}

\section{Velocidad del viento}
\subsection{Modelo Matemático} \label{ss:Modelo_Mat}
Como se adelanta anteriormente, para encontrar los parámetros de la distribución de Weibull que se ajusten a los datos de prueba se utilizará el \emph{particle swarm optimization}. La función de distribución de Weibull está definida en la ecuación \ref{eq:weibull}. El PSO a utilizar está representado por la ecuación \ref{eq:PSO}. La función objetivo se describe con la fórmula \ref{eq:PSO_FO} y es aquella con la que se busca minimizar el error cuadrático entre la frecuencia real de los datos y la estimada por la distribución de Weibull. Los parámetros a encontrar $k$ y $c$ deben ser $\geq 0$ . A modo de evitar la convergencia prematura, se utilizará la recomendación de Chang \cite{Chang10_2} para la variación de parámetros del enjambre:
\begin{align}\label{eq:VariationParameters}
    w(j) &= (1 - \frac{j}{iter_{max}})^{\alpha}(w_{max} - w_{min} + w_{min})\\
    c_{1}(j) &= (1 - \frac{j}{iter_{max}})^{\beta}(c_{1max} - c_{1min}) + c_{1min}\\
    c_{2}(j) &= (1 - \frac{j}{iter_{max}})^{\gamma}(c_{2min} - c_{2max}) + c_{2max}
\end{align}    
Donde $w_{max} = 0.9$ y $w_{min} = 0.4$, $c_{1max}$, $c_{2max}$ y $c_{1min}$, $c_{2min}$ son 2.5 y 0 respectivamente. Los parámetros $\alpha, \beta, \gamma$ son definidos como 0.5, 1.5 y 1.0 respectivamente. $iter_{max}$ es el máximo número de iteraciones.

\subsection{Representación}
Cada vector posición de las partículas del enjambre representa una solución candidata la cual varía dentro de cierto espacio de búsqueda definido por los límites de las componentes. Así, para el caso de los parámetros de la distribución de Weibull, la posición de las partículas está representada por los parámetros $k$ y $c$ quedando de la forma:
\begin{align}
    x = (k, c)
\end{align}    
Para ambos parámetros, se establecen los límites entre $0 \leq (k,c) \leq 20$, criterio que se basa en el trabajo de Carneiro et al. \cite{Carneiro15}. De esta forma, se espera que el rango sea suf, la cual representa el error de la predicción de Weibull versus los datos reales.\\
Así, para cada partícula se define una estructura que posee las siguientes propiedades: 
\begin{enumerate}\label{rep:Particle}
    \item Posición: vector de números flotantes de largo dos, los cuales representan la ubicación de la partícula dentro del espacio de búsqueda y sus componentes a los parámetros $k$ y $c$.
    \item Velocidad: vector de números flotantes que representan el cambio de valor de cada componente de la posición de la partícula en determinada iteración. Se actualiza en base a la ecuación de velocidad del PSO.
    \item Mejor resultado personal: vector flotante que guarda la mejor posición conseguida por la partícula durante las iteraciones transcurridas.    
\end{enumerate}        
Mientras que el enjambre, siendo esencialmente una estructura que posee referencia a todas las partículas, queda representado de la siguiente forma:
\begin{enumerate}\label{rep:Swarm}
    \item Partículas: Arreglo de referencias a las estructuras de partículas creadas.
    \item Mejor posición global: De todos los mejores resultados particulares a cada partícula, se almacena la mejor posición de todas. La que persiste al final del ciclo de iteraciones, es la solución final.
    \item W, C1 y C2: Son los parámetros de inercia, cognitivo y social respectivamente.    
\end{enumerate} 

\subsection{Descripción del algoritmo}
La lógica del algoritmo \ref{alg:pso} se basa en mover las partículas dentro del rango definido para los componentes de la solución hasta que todas las partículas se concentren en alguna zona que represente una buena solución al problema, no necesariamente el óptimo. Lo importante en cada iteración es actualizar o mover el enjambre, revisar y guardar las mejores soluciones y actualizar los parámetros de inercia, cognitivo y social que definen las velocidades.
%!TEX root = main.tex

\begin{algorithm}[h!]
\caption{PSO para el ajuste de los parámetros de la distribución de Weibull}
\label{alg:pso}
\begin{algorithmic}
\REQUIRE Datos de frecuencias de velocidades del viento.
\ENSURE Valores para los parámetros $k$ y $c$.
\STATE enjambre = inicializar(w,c1,c2)
\FOR{$i = 1$ to $Iter_{max}$}
\FOR{Each partículas en enjambre}
    \STATE actualizarVelocidadPartícula(partícula)
    \STATE actualizarPosiciónPartícula(partítucla)
    \STATE guardarMejorResultadoPartícula(partícula)
\ENDFOR
\STATE guardarMejorResultadoGlobal(enjambre)
\STATE actualizarParámetros(enjambre)
\ENDFOR
\STATE retornarMejorResultadoGlobal(enjambre).
\end{algorithmic}
\end{algorithm}

Las iteraciones fueron probadas hasta un máximo de 1000 y 50 partículas, (A excepción del experimento donde se consideraron todos los promedios diarios, 2013, 2014 y 2015, en el cual, se utilizaron 200 partículas). Los parámetros de $w$, $c1$ y $c2$ fueron definidos tal y como explica en el modelo matemático, en la sección \ref{ss:Modelo_Mat}.\\

\subsection{Experimentos}\label{sec:Experimentos_velocidad}
Los experimentos fueron realizados con datos del viento obtenidos por la Armada de Chile para la región de Valparaíso en los años 2013, 2014 y 2015. Estos fueron tratados mediante \emph{scripts} desarrollados en python para obtener las frecuencias de las distintas velocidades del viento registradas a lo largo del año. Los datos se organizaban de la siguiente forma: Por cada año, se tiene una tabla en un archivo excel de cada mes, en donde se registra por cada fila los resultados de la medición de cada día. Las mediciones son registradas en un intervalo de tres horas, es decir, se tienen registros diarios para las 3:00, 6:00, 9:00, 12:00, 15:00, 18:00, 21:00 y 00:00 horas.\\
Un ejemplo es la tabla mostrada en la figura \ref{fig:example_data}.
 \begin{figure}[h!]
    \centering
    \includegraphics[height=100mm]{figures/example_data.png}
    \caption{Ejemplo colección de datos Enero Valparaíso 2015}
    \vspace{-.25cm}
    \caption*{Obtenido desde el Instituto Meteorológico de la Armada de Chile.}
    \label{fig:example_data}
 \end{figure}
El ajuste de la distribución de Weibull a estos datos de velocidad del viento se hizo considerando las siguientes configuraciones para el cálculo del
histograma de frecuencias:
\begin{enumerate}
	\item \textbf{Todos los años}. Se considera el promedio diario de velocidad del viento como dato unitario para el cálculo de las frecuencias, considerando
	todos los días en el intervalo de Enero del 2013 hasta Diciembre del 2015.
	\item \textbf{Anual}. Se considera el promedio diario de velocidad del viento como dato unitario para el cálculo de las frecuencias en
	un lapso anual (2013, 2014 y 2015).
	\item \textbf{Por temporada}. Se considera el promedio diario de velocidad del viento como dato unitario para el cálculo de las frecuencias en
	un lapso de tres meses (Enero - Marzo ; Abril - Junio; Julio - Septiembre; Octubre - Diciembre).
	\item \textbf{Datos brutos}. Se considera cada medición realizada (8 por día) como dato unitario, en un lapso de un año.
\end{enumerate}

Una vez obtenido los datos de frecuencias, se procede a aplicar el algoritmo \emph{particle swarm optimization} donde se obtienen los parámetros de ajuste $k$ y $c$. De esta manera, se evalúa la calidad del modelo generado (distribución de densidad de probabilidad de Weibull), para las distintas configuraciones mediante gráficos y los siguientes test estadísticos (utilizados también en el trabajo de Carneiro et al. \cite{Carneiro15}):
\begin{enumerate}
    \item \emph{Root Mean Square Error}
        \begin{align}
            RMSE = \sqrt{\frac{\sum_{i=1}^{N}(X_i - Y_i)^2}{N}}
        \end{align}    
    \item \emph{Correlation}
        \begin{align}
            r = \frac{\sum_{i=1}^{N}(X_i - X_{med})\cdot(Y_i - Y_{med})}{\sqrt{\sum_{i=1}^{N}(X_i - X_{med})^2}\cdot\sqrt{\sum_{i=1}^{N}(Y_i - Y_{med})^2}}
        \end{align}    
    \item \emph{Relative Bias}
        \begin{align}
            RB = \frac{X_{med} - Y_{med}}{Y_{med}}  
        \end{align}    
\end{enumerate}        
Donde $N$ es el número de datos, $Y_i$ la frecuencia de dichos datos, $X_i$ la frecuencia entregada por la distribución de Weibull, $X_{med}$ la media de $X_i$ e $Y_{med}$ la media de $Y_i$. \\

Las pruebas fueron realizadas en un computador con sistema operativo Ubuntu 16.04 64-bit, 3.8 GB de memoria y procesador doble núcleo Intel Pentium 2.60 GHz. 

\section{Dirección del viento}
\subsection{Modelo Matemático}\label{ss:model_math_dir} 
Como se comenta anteriormente, la distribución de densidad de probabilidad que se utilizará para describir la distribución de datos de dirección del viento
es la \emph{finite mixtures of von mises distribution} descrita en \ref{eq:mixtureVonMises}, la cual consiste básicamente en una combinación lineal de la \emph{simple von Mises distribution} descrita en \ref{eq:simpleVonMises}.\\ 
De forma preliminar, los datos se ordenan en un histograma de densidad con el cual se obtiene un esqueleto de la distribución de densidad de probabilidad. Posteriormente se requieren encontrar los parámetros de ajuste $\mu_j$, $k_j$ y $w_j$ para cada $j$-ésima \emph{simple von Mises distribution}. La forma en que
se realiza esto último en este trabajo está basado en el documento de Carta et al. \cite{Carta07} y se describe a continuación.\\
Para la construcción del histograma se divide el rango de datos que va de 0 a $2\pi$ en $T$ clases con frecuencia $O_i$ la cual representa la suma de las observaciones en el rango de la clase $T$. Posteriormente se definen $k$ sectores del mismo largo desde las $T$ clases, relacionados al número de direcciones de viento predominantes (o con mayor frecuencia). Esto define el número de funciones de von Mises a utilizar. La estimación de $k$ se realiza mediante la observación del histograma generado, un análisis cualitativo de las direcciones predominantes en los datos. También, se sigue la observación empírica en el trabajo de Carta et al. \cite{Carta07}, en donde se concluye que valores superiores a 6 \emph{mixtures} no mejoran considerablemente la calidad del ajuste.\\
Para la aproximación inicial de los parámetros de la \emph{mixture of von mises distribution} se utiliza una estimación numérica, utilizada por Heckenbergerova et al. \cite{Heckenbergerova15} \cite{Heckenbergerova13} y Carta et al. \cite{Carta07}, basada en los datos recolectados acerca de la dirección del viento.\\
Sea $j \in \{1 ... k\}$ el subíndice del sector representado por la $j$-ésima función de von Mises.\\
La dirección del viento predominante $\mu_j$ se estima de la siguiente forma:
\begin{align}\label{eq:Prevailing_Param}
    \mu_j &= 
        \left\{
            \begin{array}{ll}
                arctan(\frac{s_j}{c_j})  & s_j \geq 0, c_j > 0\\
                \frac{\pi}{2} & s_j > 0, c_j = 0\\
                \pi + arctan(\frac{s_j}{c_j}) & c_j < 0\\
                \pi & s_j > 0, c_j = -1\\
                2\pi + arctan(\frac{s_j}{c_j}) & s_j < 0, c_j > 0\\
                3\frac{\pi}{2} & s_j < 0, c_j = 0\\
            \end{array}
        \right.
\end{align}
En donde $s_j$ y $c_j$ representan el seno y coseno promedio del sector $j$.\\
Tradicionalmente, se estima el parámetro de concentración $k_j$ con la ecuación:
\begin{align}\label{eq:Implicit_Param}
    \frac{I_1(k_j)}{I_0(k_j)} = \sqrt{s_j^2 + c_j^2}
\end{align}
Donde $I_1(k_j)$ es la función modificada de Bessel de primera clase y orden 1.
 Como se explica en Banerjee et al. \cite{Banerjee05}, debido a la falta de una solución análitica a la ecuación \ref{eq:Implicit_Param}, no es posible estimar directamente los valores de $k$. Se podrían utilizar métodos para ecuaciones no lineales, pero para datos de altas dimensiones, problemas de desbordamiento (\emph{overflow}) o inestabilidad numérica se vuelven concurrentes. Por tanto, se utiliza la propuesta realizada en el trabajo de Heckenbergerova et al. \cite{Heckenbergerova15} con lo cual el parámetro $k_j$ puede ser aproximado por:\\
\begin{align}
    |k_j| = \{23.29041409 - 16.8617370\sqrt[4]{s_j^2 + c_j^2}\} 
\end{align}
También existe otra forma similar a esta aproximación, utilizada por  Heckenbergerova et al. en \cite{Heckenbergerova13} y por Carta et al. \cite{Carta07}, la cual consiste en 
la siguiente fórmula:\\
\begin{align}
    k_j = \{23.29041409 - 16.8617370\sqrt[4]{s_j^2 + c_j^2} - 17.4749884 exp(-(s_j^2 + c_j^2)\}^-1 
\end{align}
Los pesos iniciales $w_j$ son aproximados como: \\
\begin{align}\label{eq:Weight_Param}
    w_j = \frac{\sum_{i=J_l}^{J_u} O_i}{\sum_{i=1}^{T} O_i}
\end{align}

Donde $J_l$ y $J_u$ son los índices de los bordes del sector $j$.\\
Esta estimación inicial de los parámetros de la \emph{mixture of von mises distribution} es mejorada mediante la meta-heurística \emph{particle swarm optimization}. Para el PSO se utiliza la representación descrita en \ref{eq:PSO}. En el trabajo de Heckenbergerova et al. \cite{Heckenbergerova15} se fijan los parámetros del PSO. Para la inercia $w = 0.89$, para el parámetro cognitivo $c_1 = 0.5$ y para el parámetro social $c_2 = 0.7$. Sin embargo, estos valores no generan resultados satisfactorios, como se concluyó además en su trabajo, los valores de la función objetivo no satisfacen los requerimientos para la hipótesis impuesta en el test realizado. Por ello, en este trabajo se propone una forma de hacer variar los parámetros del PSO durante la ejecución.\\ 
En la sección anterior se menciona la estrategia sugerida en el trabajo de Chang \cite{Chang10_2} para la variación de parámetros, descrita en \label{eq:VariationParameters}, con el fin de evitar una convergencia prematura para el ajuste de la distribución de Weibull. Sin embargo, esta sugerencia presenta inconvenientes por que depende del número de iteraciones del algoritmo. A pesar de que en el caso del ajuste de la distribución de Weibull no hubo mayores inconvenientes (debido a que los tiempos eran menores a 1 segundo, y por tanto, el número de iteraciones no era un factor importante), para el ajuste de la distribución de von Mises, el número de iteraciones del algoritmo si es un parámetro importante para el rendimiento del método. Esto, debido a que la complejidad de la solución aumenta exponencialmente. Mientras que para la distribución de Weibull, la dimensión de a solución era 2, para la \emph{mixture of von Mises distribution}, la dimensión es de $3N$, donde $N$ es el número de mezclas elegidas.\\
Como el número de iteraciones afecta considerablemente el rendimiento del algoritmo, al realizar diversas pruebas con cantidad de iteraciones distintas, se pudo evidenciar que el hecho de que las iteraciones sean un valor que afecta la variación de parámetros en la fórmula propuesta por Chang \label{eq:VariationParameters}, tiene como implicancia que el algoritmo tome diferentes caminos hacia la solución final dependiendo del valor definido y por ende, una aumento en el número de iteraciones no constituiría una mejora en la solución, lo cual, es contraproducente con lo que se espera de la meta-heurística, la cual debe permitir un ajuste entre costo computacional (tiempo de ejecución) y calidad de la solución encontrada.\\
Por lo argumentado anteriormente, se propone el siguiente ajuste de parámetros inspirado en lo propuesto por Chang \cite{Chang10_2}.\\  
\begin{align}\label{eq:VariationParameters_new}
    w(j) &= w_j + (w_{max} - w_j) * F \\
    c_{1}(j) &= c_1 + (c_{1max} - c_1) * F \\
    c_{2}(j) &= c_2 + (c_{2min} - c_2) * F
\end{align} 
Donde $F$ es un factor de avance dentro de un rango definido para los parámetros el cual está en función de una valor de la función objetivo esperado y el valor actual de la solución. Es decir: $F = \frac{Valor esperado en la FO}{Valor actual en la FO}$. Si el factor sobrepasa el valor de 1 (caso en el que la solución actual es mejor que la esperada), se deja de cambiar los parámetros. Esto se resume como:
\begin{align}\label{eq:restriction_var_par_new}
    F &= 
        \left\{
            \begin{array}{ll}
                \frac{P}{p_j}  & P \leq p_j\\
                0 & P > p_j\\
            \end{array}
        \right.
\end{align}
Donde $P$ es un valor esperado para la función objetivo (definido como parámetro) y $p_j$ es el valor en la función objetivo de la mejor solución hasta el momento.
 Con esto se logra mejorar el rendimiento del algoritmo alcanzando los valores deseados en tiempo menores como se puede revisar en la sección de resultados.\\
La función objetivo para el PSO es el test estadístico $\chi^2$ descrito en Heckenbergerova et al. \cite{Heckenbergerova15} como sigue a continuación:
\begin{align}\label{eq:FO_Direction}
    \chi^2 = \sum_{i=1}^{T}\frac{(O_i - np_{i})^2}{np_i}
\end{align}
Donde $T$ es el número de clases de frecuencia definido para construir el histograma, $n$ es la suma de las frecuencias observadas $O_i$ y $p_i$ es la probabilidad teórica de cada clase de frecuencia predicha por el modelo ajustado.\\
Para el cálculo del $p_i$ se utiliza:
\begin{align}
    p_i = \int_{l_i}^{u_i} f(x) dx
\end{align}
Donde $u_i$ y $l_i$ son los bordes de la $i$-ésima clase de frecuencia.\\
La forma de la solución a encontrar es descrita en \ref{eq:sol_pso}. Esta es restringida por la condición para los pesos de la \emph{mixture von mises distribution}, la cual obliga a que se deba cumplir que la suma de los pesos sea igual a 1, como se describe en \ref{eq:WeightConstraint}.

\subsection{Representación}\label{sec:Representacion}
La representación del PSO es similar al utilizado para el ajuste de la distribución de datos de velocidad del viento. Las partículas y el enjambre están representados por \ref{rep:Particle} y \ref{rep:Swarm} respectivamente.\\
La solución para el PSO que mejora la estimación inicial de los parámetros para la \emph{mixture von Mises distribution} está representado por un vector $v$ en el cual se encuentran los valores para todos los parámetros de cada \emph{simple von Mises distribution}. Estos valores están codificados para que el algoritmo se mueve en el rango desde 0 a 1.\\
El vector solución tiene la forma:
\begin{align}
    v = (\overbrace{v_1,...,v_k}^{\mu},\overbrace{v_{k+1},...,v_{2k}}^{k},\overbrace{v_{2k+1},...,v_{n}}^{w}).
\end{align}
El parámetro $\mu$ está representado en el rango $i \in \{1,...,k\}$ y para ser decodificado debe ser escalado por $2\pi$.\\
El parámetro $k$ está representado en el rango $i \in \{k+1,...,2k\}$ y para ser decodificado debe ser escalado por $[0, 700]]$.\\
El parámetro $w_j$ está representado en el rango $i \in \{2k+1,...,n\}$ cuyos valores van en el rango $[0,1]$.

\subsection{Descripción del algoritmo}
El algoritmo para el ajuste de la función de densidad de probabilidad para la dirección del viento se basa en la propuesta de Heckenbergerova et al. \cite{Heckenbergerova15}.\\ 
Como se ha ido vislumbrado, consiste en dos fases. La primera, una aproximación basada en la estimación numérica de los parámetros requeridos para la \emph{mixture of von Mises distribution} a través de operaciones simples con los datos recolectados, y la segunda, una mejora de la solución inicial obtenida en la fase anterior mediante el uso de la meta-heurística \emph{particle swarm optimization}. \\
El algoritmo para la aproximación inicial de la solución se describe en \ref{alg:init_aprox_direction}.
%!TEX root = main.tex

\begin{algorithm}[h!]
\caption{Aproximación inicial de los parámetros de la \emph{mixture von Mises distribution}}
\label{alg:init_aprox_direction}
\begin{algorithmic}
\REQUIRE Datos de frecuencias de la dirección del viento.
\REQUIRE K, Cantidad de \emph{simple von Mises distribution}.
\REQUIRE T, clases de frecuencias.
\REQUIRE D, Total de datos.
\ENSURE Valores para los parámetros $\mu_j$, $k_j$ y $w_j$, para cada $j \in \{1,...,k\}$.
\STATE sol = inicializarVectorSolución(3*K)
\FOR{$j = 0$ to $K$}

\STATE datos$_j$ = datosEnRango($j*D/K$)
\STATE s$_j$ = obtenerSenoPromedio(datos$_j$)
\STATE c$_j$ = obtenerCosenoPromedio(datos$_j$)
\STATE u$_j$ = obtenerDirecciónPredominante(s$_j$,c$_j$)
\STATE k$_j$ = obtenerConcentración(s$_j$, c$_j$)
\STATE w$_j$ = obtenerPeso($j*(T/K)$, $(j + 1)*(T/K)$)
\STATE addToSolution(sol, u$_j$, k$_j$, w$_j$)

\ENDFOR
\STATE retornarSoluciónInicial(sol).
\end{algorithmic}
\end{algorithm}
En donde la estimación de los parámetros se realiza como se describe en \ref{eq:Prevailing_Param} para los $\mu_j$, \ref{eq:Implicit_Param} para los $k_j$ y \ref{eq:Weight_Param} para los pesos $w_j$.
Una vez obtenida la aproximación inicial se procede a mejorar esta mediante el uso del PSO  \ref{alg:pso_direction}.
%!TEX root = main.tex
\begin{algorithm}[h!]
\caption{PSO para la mejora de la aproximación de los parámetros de la \emph{mixture von Mises distribution}}
\label{alg:pso_direction}
\begin{algorithmic}
\REQUIRE Datos de la dirección del viento.
\REQUIRE Solución inicial para el ajuste de la \emph{mixture von Mises distribution}.
\ENSURE Solución inicial mejorada.
\STATE enjambre = inicializar(w,c1,c2)
\FOR{$i = 1$ to $Iter_{max}$}
\FOR{Each partículas en enjambre}
    \STATE actualizarVelocidadPartícula(partícula)
    \STATE actualizarPosiciónPartícula(partítcula)
    \STATE revisarLímitesPosición(partícula)
    \STATE guardarMejorResultadoPartícula(partícula)
\ENDFOR
\STATE guardarMejorResultadoGlobal(enjambre)
\STATE actualizarParámetros(enjambre)
\ENDFOR
\STATE retornarMejorResultadoGlobal(enjambre).
\end{algorithmic}
\end{algorithm}
El algoritmo es bastante similar en estructura al desarrollado para la velocidad del viento \ref{alg:pso}. Sin embargo,
existen diferencias relevantes al problema actual que se destacarán a continuación.\\
Para la inicialización de las partículas, se realizaron pequeñas perturbaciones a la solución inicial tal y como se sugiere en Heckenbergerova et al. \cite{Heckenbergerova15}. Esto evita que la solución escape a zonas que tengan un buen resultado en la función objetivo, pero que la forma escape a la del histograma. Debido a que la función objetivo definida \ref{eq:FO_Direction} mide las diferencias de frecuencias entre los datos reales y los teóricos, es decir, las áreas de las barras del histograma de densidad versus el área bajo la curva de la distribución de probabilidad en algún intervalo, más de una forma de la curva podría parecer una buena solución como se vé en el gráfico de muestra de un ajuste no deseado \label{fig:pso_valpo_15_14_13_lq}. Por ello, la idea es mantener la forma inicial encontrada, mejorándola sin deformarla. Así, las perturbaciones iniciales a los valores de las posiciones de las partículas eran del orden de $~ 10^{-3}$.\\
La forma en que se cuidaron las condiciones de borde consistieron en limitar el avance de las partículas a los bordes 0 y 1 manteniéndolos en dichos valores si es que se excedían a ellos.\\
Para cuidar la restricción de pesos se normalizaran los valores determinados en cada iteración, es decir, se suman todos los valores $w_j$ y se ponderan dichos valores por el recíproco de la suma obtenida.\\
Debido a que la función objetivo implica determinar la frecuencia teórica, es necesario determinar la probabilidad
de cierto rango de direcciones mediante el cálculo del área bajo la curva de la distribución de densidad de probabilidad para luego ser multiplicada por la suma del total de datos y así obtener el valor requerido. Por ende, para el cálculo de la integral se utilizaron sumas de Riemann con una partición conveniente al desempeño del algoritmo y la precisión requerida.\\
Finalmente, la solución obtenida es decodificada tal y como se explica en la sección anterior \ref{sec:Representacion}.\\

\subsection{Experimentos}
Similar a los descrito en \label{sec:Experimentos_velocidad}, los datos de dirección del viento son tratados para rescatar las mediciones pertinentes al trabajo aquí expuesto. Estos se encuentran inicialmente en un formato como el que se puede apreciar en la figura \ref{fig:example_data}.\\
Nuevamente, las pruebas fueron realizadas en un computador con sistema operativo Ubuntu 16.04 64-bit, 3.8 GB de memoria y procesador doble núcleo Intel Pentium 2.60 GHz.\\
Para evaluar la calidad de la solución, se utiliza el test \emph{Chi square goodness fit}\cite{goodFitTest}, con lo cual se evalúa que tan bien representa el modelo propuesto a los datos medidos. Para ello, la hipótesis nula $H_0$ es que los datos de dirección del viento se distribuyen según la función \emph{mixture of von Mises distribution} y la hipótesis alternativa $H_1$ niega dicha afirmación. Se rechaza $H_0$ si el valor de la función objetivo del PSO para la solución final encontrada excede el valor crítico de $\Xi^2$ para un nivel de significancia de $\alpha = 0.05$ y 13 grados de libertad, es decir \textbf{22.362}, el cual se puede encontrar en la tabla de la distribución $\Xi^2$ \cite{chiSquareTable}. Los grados de libertad son definidos a partir de la cantidad de clases de frecuencia definidas para el estudio, en este caso, se dividió el rango de valores de $[0, 2\pi]$ en 14 tramos iguales, por lo quedan $(n-1)$ grados de libertad, 13 en este caso.\\
Los experimentos consistieron en el ajuste de varios subconjunto de datos provenientes de las mediciones obtenidas para la dirección del viento en los años 2013, 2014 y 2015. Así, se prueba la utilidad de la propuesta realizada independiente del rango de tiempo a modelar.\\
Es importante considerar que no se incluyen los días en los que no hubo viento por la evidente imposibilidad de registrar la dirección.
Los subconjuntos definidos fueron los siguientes:
\begin{enumerate}
    \item \textbf{Anual}: Se consideran los datos de todo el año elegido.
    \item \textbf{Meses acumulados}: Se considera una agrupación mensual pero reuniendo los datos de tres años consecutivos (2013, 2014, 2015). Es decir, para el mes de Enero, se ajusta el modelo a los datos de Enero-2013, Enero-2014 y Enero-2015 en conjunto.
    \item \textbf{Meses}: Se escogen algunos meses para ser comparados consigo mismos durante los tres años escogidos. Por ejemplo, Enero-2013, Enero-2014 y Enero-2015 por separado.       
\end{enumerate}

Para el funcionamiento del PSO, se estableció un límite de 50500 iteraciones, se utilizaron 100 partículas, y se usó un límite de parada si es que el valor en la función objetivo de la mejor solución encontrada al momento era menor a 22.362 (criterio basado en la estrategia de cumplir el test \emph{Chi square goodness fit}). La cantidad de \emph{mixture of simple von Mises distribution} utilizadas fue de 7, siguiendo la recomendación de Carta et al. \cite{Carta07} y algunas pruebas iniciales.\\

Por último, se comparan los resultados obtenidos por el PSO propuesto en el trabajo de Heckenbergerova et al. \cite{Heckenbergerova15} con la propuesta realizada en este trabajo 
\ref{eq:VariationParameters_new}.

\section{Uso de los algoritmos propuestos}
El esquema \ref{esq:PSO_ALG} resume el funcionamiento del algoritmo para el ajuste del modelo probabilístico a los datos del viento. A continuación se detallarán los pasos a seguir en el proceso de obtener el ajuste.
\begin{enumerate}
    \item Lo primero que se realiza es el formateo de los datos, es decir, la conversión desde un set de datos externos a el formato utilizado por el programa desarrollado, CSV. El \emph{script} utilizado dependerá del formato externo en el cual se proveen los datos.
    \item Una vez obtenido los datos en el formato deseado, se procede a calcular las frecuencias de los registros obtenidos, para posteriormente ser comparadas con el modelo teórico.
    \item Antes de comenzar con el ajuste, se calcula una solución inicial ya sea de forma aleatoria (como es el caso para el ajuste de la velocidad del viento) o mediante algún método conveniente (como la aproximación numérica para el caso del ajuste de la dirección del viento). La elección de esta última implica que se quiere guiar la solución a encontrar.  
    \item Definiendo la cantidad de partículas y el número de iteraciones, se ejecuta el algoritmo \emph{particle swarm optimization} para que mejore la estimación inicial de acuerdo a una función objetivo previamente definida. 
    \item Finalmente, una vez que el algoritmo termina, entrega los valores de los parámetros para la función de densidad de probabilidad (fdp). Con ello, es posible elaborar un histograma y graficar la fdp para evaluar cualitativamente el ajuste obtenido.
\end{enumerate}
Una vez finalizado el proceso, se tiene un modelo matemático con el cual es posible trabajar para, por ejemplo, alguna de las aplicaciones que se exponen en la siguiente sección de análisis.\\
En meteorología, la velocidad del viento se registra en nudos por segundo, pero en este trabajo se convirtieron los datos al SI, es decir, a metros por segundo. La conversión es sencilla, basta multiplicar el valor de los nudos/segundo por 0.514444 para obtener los metros/segundo.\\
El sistema de referencia utilizado para la interpretación de los datos de dirección del viento es el conocido como la rosa de los vientos \cite{RosaViento}. Dicho sistema ubica el norte en el grado 0, al este en el grado 90, al sur en el grado 180 y al oeste en el grado 270, siendo la dirección marcada el origen desde donde proviene el viento. Ejemplo, si la dirección predominante del viento en cierto intervalo de tiempo es de 90 grados, entonces diremos que la corriente de viento proviene desde el este.

\begin{figure}[ht!]
\caption{Esquema de uso del algoritmo}
% Define block styles
\tikzstyle{decision} = [diamond, draw, fill=blue!20, 
    text width=4.5em, text badly centered, node distance=3cm, inner sep=0pt]
\tikzstyle{block} = [rectangle, draw, fill=blue!20, 
    text width=15em, text centered, rounded corners, minimum height=4em]
\tikzstyle{blockALG} = [rectangle, draw, fill=green!20, 
    text width=15em, text centered, rounded corners, minimum height=4em]
\tikzstyle{line} = [draw, -latex']
\tikzstyle{cloud} = [draw, ellipse,fill=red!20, node distance=7cm,
    minimum height=2em]
    
\begin{tikzpicture}[node distance = 3cm, auto]
    % Place nodes
    \node [block] (dataFormat) {1. Formateo de los datos};
    \node [cloud, left of=dataFormat] (inputData) {Datos};
    %\node [cloud, right of=init] (system) {system};
    \node [blockALG, below of=dataFormat] (initAlg) {2. Cálculo de frecuencias};
    \node [blockALG, below of=initAlg] (SolIni) {3. Estimación inicial};
    \node [blockALG, below of=SolIni] (PSO) {4. PSO mejora la estimación inicial};
    \node [block, below of=PSO, node distance=3cm] (grafico) {5. Visualización con histograma y la fdp};
    % Draw edges
  
    \path [line] (dataFormat) -- node {\footnotesize{\emph{CSV con datos formateados}}}(initAlg);
    \path [line] (initAlg) -- node {\footnotesize{\emph{Fecuencias calculadas, parámtros}}}(SolIni);
    \path [line] (SolIni) -- node {\footnotesize{\emph{Estimación inicial}}}(PSO);
    \path [line] (PSO) -- node {\footnotesize{\emph{Vector de parámetros}}}(grafico);
    \path [line,dashed] (inputData) -- (dataFormat);
\end{tikzpicture}

\label{esq:PSO_ALG}
\end{figure}